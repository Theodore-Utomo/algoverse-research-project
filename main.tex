\documentclass[11pt]{article}
\usepackage[margin=1in]{geometry}
\usepackage{amsmath}
\usepackage{graphicx}
\usepackage{hyperref}

\title{<Research Title>}
\author{Theodore Utomo, Kometh, Aritro, Dev}
\date{\today}

\begin{document}

\maketitle

\section{Introduction}
Understanding the spatial relationship between healthcare centers and the populations they serve is crucial for optimizing access to urgent care. In this study, we develop a spatio-temporal database incorporating population density, settlement locations, healthcare centers, and travel factors in Kenya to assess accessibility and inform infrastructure planning.

\section{Relevant Past Papers}
For each paper, consider including:
\begin{itemize}
    \item One sentence summary of the paper.
    \item Gap/limitation of the paper.
\end{itemize}
The main focus of our experiment is to see how accessible it is for Kenyan people to access healthcare. Our idea is to map geospatial population data along with data for Kenyan healthcare centers to see the average or maximum time it would take for someone to commute to or from a healthcare center. Relevant research has been conducted in a similar manner to map the commute time to education centers such as primary schools. The techniques used take into account the nuances of travel time such as walking, traffic congestion, difficult roads, etc. (see \url{https://www.tandfonline.com/doi/full/10.1080/14733285.2022.2137388#d1e808}).

\section{Motivation} 
\begin{itemize} 
\item \textbf{Problem Statement:} In many developing countries, including Kenya, infrastructural limitations hinder access to healthcare, adversely affecting patient outcomes. 
\item \textbf{Significance:} Ensuring timely access to healthcare is essential, especially for low-income and rural communities, as delays can result in poor health outcomes. 
\item \textbf{Proposed Approach:} We leverage existing datasets—population density, settlement locations, and healthcare center information—augmented with travel time considerations to identify accessibility gaps and pinpoint optimal locations for new infrastructure. 
\item \textbf{Rationale:} Integrating diverse geospatial datasets and modeling realistic travel conditions provides a comprehensive assessment of healthcare accessibility, which is critical for effective infrastructure planning. \end{itemize}

Kenya faces unique challenges in healthcare delivery due to its developing infrastructure and the mobility of its population, including nomadic pastoralists. While the country's population and economy are growing rapidly, infrastructural deficits continue to impede effective healthcare access. By identifying potential settlement areas and quantifying travel time to healthcare facilities, our study aims to provide actionable insights for optimizing healthcare infrastructure development.

\section{Key Contributions/Ideas}
\begin{itemize}
    \item Integration of spatial population data, settlement data, and healthcare center data.
    \item Incorporation of factors affecting travel time (e.g., road quality, congestion, age of commuters).
    \item Development of an accessibility model that suggests optimal locations for new healthcare infrastructure.
\end{itemize}

\section{Methods}
\subsection{Datasets}
We plan to gather data for three primary components:
\begin{itemize}
    \item \textbf{Spatial Population/Population Density:} We obtained our estimated population density map through GRID3. This dataset is a heat map of Kenya with data collected at a spatial resolution of 3 arc-seconds, providing population estimates for each 100m x 100m grid cell. Metadata, including age, is provided to account for varying travel speeds.
    \item \textbf{Healthcare Infrastructure:} Our database of healthcare facilities comes from OpenStreetMap’s Kenya Health Facilities dataset, which includes details on location, amenity type, service type, capacity, etc.
    \item \textbf{Factors Affecting Travel:} Additional datasets will be collected to account for factors such as road networks (available via OpenStreetMap shapefiles), traffic congestion, transportation methods, and the age of commuters.
\end{itemize}

\subsection{Travel Time Modeling}
Once all input layers are gathered, we will use AccessMod’s merging tools to combine the data and conduct an analysis of health coverage. Our model will compute the average travel time to the nearest healthcare center while considering various transportation challenges.

\subsection{AccessMod Analysis Features}
After computing the travel times, we can use AccessMod's several analytical features including a scaling up feature that identifies optimum locations for constructing new healthcare centers.

\subsection{CNN Model For Identifying Optimal Types Of Health Care Facilities}
Once we have AccessMod's suggestions for optimum locations to create new health care facilities, we can leverage deep learning models to identify the optimal types of health care facilities to construct. We can execute this by fine tuning existing CNN models to identify already existing health care infrastructures to allow it to identify optimal existing health care structures in relation to the overall population. We can then use unsupervised learning methods to identify the types of health care infrastructure that will be optimal to build in the locations suggested by AccessMod. 

\section{Evaluation and Metrics}
To assess the effectiveness of our proposed method, we will evaluate the improvements in healthcare accessibility using several quantitative metrics. Our evaluation metrics include:
\begin{itemize}
    \item \textbf{Commute Time Reduction:} Measure the reduction in both average and maximum travel times to the nearest healthcare facility after the implementation of our suggested infrastructure improvements.
    \item \textbf{Accessibility Improvement:} Compare the healthcare accessibility scores (e.g., coverage indices) before and after the proposed changes against established benchmarks.
    \item \textbf{Model Performance Metrics:} For the CNN component that identifies optimal types of healthcare facilities, we will use standard classification metrics such as accuracy, precision, recall, and F1-score on a validation set, in addition to confusion matrices to evaluate class-wise performance.
    \item \textbf{Ablation Studies:} Systematically disable or modify parts of our model (e.g., using or excluding specific travel factors) to isolate and quantify the impact of each component on the overall system performance.
\end{itemize}
In addition to these metrics, we will perform sensitivity analyses to assess how changes in input data quality and travel time modeling assumptions affect the overall results.

\section{Experimental Setup}
Our experimental setup is divided into several stages to systematically evaluate and validate the proposed methodology:

\begin{enumerate}
    \item \textbf{Data Collection and Preprocessing:}
    \begin{itemize}
        \item \textbf{Population Data:} Acquire high-resolution population density maps from GRID3, ensuring that spatial resolution and demographic metadata (e.g., age distribution) are retained.
        \item \textbf{Healthcare Facilities:} Gather healthcare facility data from OpenStreetMap’s Kenya Health Facilities dataset, including geolocations and facility attributes.
        \item \textbf{Travel Factors:} Collect additional geospatial layers (e.g., road networks from OpenStreetMap shapefiles, traffic congestion data, and public transit routes) to accurately model travel times.
        \item \textbf{Preprocessing:} Normalize and georeference all datasets so they can be integrated into a unified spatial framework.
    \end{itemize}

    \item \textbf{Geospatial Analysis using AccessMod:}
    \begin{itemize}
        \item Utilize AccessMod to compute baseline travel times to existing healthcare facilities, accounting for terrain, road quality, and other travel-affecting factors.
        \item Leverage AccessMod’s scaling and optimization features to generate initial suggestions for new healthcare facility locations based on minimizing travel times.
    \end{itemize}

    \item \textbf{CNN Model Training for Facility Type Identification:}
    \begin{itemize}
        \item \textbf{Dataset Preparation:} Assemble a labeled dataset of existing healthcare facilities (with various types such as clinics, hospitals, etc.) using the collected data.
        \item \textbf{Model Architecture:} Fine-tune a pre-trained CNN to classify healthcare facility types, adapting the network to our specific dataset and task.
        \item \textbf{Training:} Train the CNN using standard backpropagation and optimization techniques (e.g., Adam optimizer), with a portion of the data reserved for validation.
        \item \textbf{Unsupervised Analysis:} Apply unsupervised learning methods (e.g., clustering) on the features extracted by the CNN to determine the optimal facility types that should be constructed in the proposed new locations.
    \end{itemize}

    \item \textbf{Integration and End-to-End Evaluation:}
    \begin{itemize}
        \item Combine the results from the AccessMod analysis and the CNN model to propose an integrated plan for healthcare infrastructure improvement.
        \item Evaluate the proposed plan using the previously defined metrics, and perform ablation studies to determine the contribution of each component.
        \item Visualize the proposed improvements using geospatial maps and statistical plots to clearly demonstrate accessibility gains.
    \end{itemize}

    \item \textbf{Benchmarking and Validation:}
    \begin{itemize}
        \item Compare our method against existing benchmarks and alternative approaches in the literature to validate the effectiveness of our approach.
        \item Conduct cross-validation and sensitivity analysis to ensure that the improvements are robust to variations in the input data and model parameters.
    \end{itemize}
\end{enumerate}


\subsection{Visualizations and Statistics}
Visual representations of population density, healthcare facility locations, and computed travel times will be used to illustrate findings. Statistical analysis will compare our model’s suggested improvements with current accessibility metrics.

\section{Potential Limitations}
While our approach integrates multiple datasets and realistic travel factors, potential limitations include:
\begin{itemize}
    \item Incomplete or outdated datasets.
    \item Challenges in accurately modeling the nuances of real-world travel conditions.
    \item Variability in data quality across different regions.
\end{itemize}

\section{Conclusion}


\bibliographystyle{plain}
\bibliography{references}

\end{document}
